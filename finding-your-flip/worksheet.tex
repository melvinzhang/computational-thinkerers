\documentclass[a4paper,12pt]{article}

\usepackage{amsfonts, amsmath}
\usepackage{exsheets}
\usepackage[a4paper, margin=1in]{geometry}

\newcommand\Blank[2][.33\linewidth]{%
  % add strut to allow for handwriting
  \rule{0pt}{4ex}%
  % Prompt
  #2\enspace
  % And the actual line
  \makebox[#1]{\hrulefill}}

\ifdefined\withsolution
    \SetupExSheets{solution/print=true,no-skip-below=true}
\fi

\title{Computational Thinkerer Workshop\\
Finding your Flip / Checking your IC number
\vspace{-2cm}}
%\author{Melvin Zhang}
\date{}

\begin{document}
\maketitle
\begin{flushright}
  \Blank{Name:}

  \Blank{Date:}
\end{flushright}

Digital information is stored and transmitted as bits, a series of
zeros and ones. Sometimes one or more bits may be flipped due to noise or damage
to the media. How can we detect and fix such errors?
\let\thefootnote\relax\footnote{Solution available at \texttt{http://melvinzh.sdf.org/2-FYF-worksheet-soln.pdf}}

\section*{Finding your Flip}
In this magic trick, we ask the audience to introduce an ``error'' and then fix
it without knowing where is the error. This is known as \textbf{error correction}.
\begin{enumerate}
    \item Ask your partner to setup a 5x5 grid of black and white stones.
    \item Add an additional column and row to make it 6x6.
    \item With your back turned, have your partner flip only one of the stones.
    \item Turn around and magically point out the stone that has just been flipped.
\end{enumerate}

\begin{question}[skip-below=3\baselineskip]
How can we figure out which stone was flipped? Hint: the additional column and row is important!
\end{question}
\begin{solution}
The additional row/column is setup so that each row/column as a even number of black stones.
When a stone is flipped, the row and column will have an odd number of black stones.
From the odd row and odd column, we can identify the exact stone that flipped.  
\end{solution}

\begin{question}[skip-below=3\baselineskip]
Practice the magic trick in pairs.
\end{question}

\begin{question}[skip-below=3\baselineskip,name=Advance exercise]
Now ask your partner to flip two stones instead of one. Can you find the two flipped stones?
\end{question}
\begin{solution}
No, we can detect the error but we cannot fix it. We can detect the error as
there is at least one odd row or odd column.  
\end{solution}

\newpage
\section*{Checking your IC number}
Ever wondered what is the meaning of the letter at the end of your NIRC/FIN
number? It serves the same purpose as the extra stones we added in
``Finding your Flip''.

By comparing the digits of your number against the letter we can check for
certain types of errors. This is known as \textbf{error detection}.

Suppose your IC number starts with S\emph{ABCDEFG}, where $A$ is the first
digit, $B$ is the second digit and so on.
\begin{enumerate}
    \item sum $= (2 \times A) +
                 (7 \times B) +
                 (6 \times C) +
                 (5 \times D) +
                 (4 \times E) +
                 (3 \times F) +
                 (2 \times G)$
    \item Add 4 to sum if your IC number starts with T or G.
    \item Find the remainder when sum is divided by 11.
    \item Convert the remainder into a letter using the following table.
\end{enumerate}

\begin{tabular}{r||ccccccccccc}
    remainder     & 0 & 1 & 2 & 3 & 4 & 5 & 6 & 7 & 8 & 9 & 10 \\ \hline
    letter (NRIC) & J & Z & I & H & G & F & E & D & C & B & A  \\
    letter (FIN)  & X & W & U & T & R & Q & P & N & M & L & K
\end{tabular}


\begin{question}[skip-below=0\baselineskip]
What is the letter for the following NRIC numbers?
\begin{itemize}
    \item S0000001? (Yusof bin Ishak, first President of Singapore)
    \item S0000002? (Wee Chong Jin, first Chief Justice of Singapore)
    \item S0000003? (Lee Kuan Yew, first Prime Minister of Singapore)
    \item S0000004? (Kwa Geok Choo, wife of Lee Kuan Yew)
    \item S0000005? (Toh Chin Chye, first Deputy Prime Minister of Singapore)
    \item S0000006? (Goh Keng Swee, second Deputy Prime Minister of Singapore)
    \item S0000007? (S Rajaratnam, first Minister for Culture)
\end{itemize}
\end{question}
\begin{solution}
I, G, E, C, A, Z, H
\end{solution}

\begin{question}[skip-below=3\baselineskip,name=Advance exercise]
Check the letter for your NRIC/FIN number. Ask your partner for
his/her NRIC/FIN number without the letter, then compute
the letter and see if you can get it right.
\end{question}

\begin{question}[skip-below=3\baselineskip,name=Advance exercise]
If we change a single digit, does the letter always change?
\end{question}

\begin{solution}
Yes! This means we can always detect an error when one of the digits is wrong.
The letter is based on the remainder and the remainder will always change as the
weights (2,7,6,5,4,3,2) do not have a common factor with 11.
\end{solution}

\begin{question}[skip-below=3\baselineskip,name=Advance exercise]
If we change \emph{two} digits, does the letter always change?
\end{question}

\begin{solution}
No. For example, swapping the first and last digit does not change the latter,
i.e. S0000001I and S1000000I are both valid IC numbers
\end{solution}

\end{document}
