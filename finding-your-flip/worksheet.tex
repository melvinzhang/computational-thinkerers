\documentclass[a4paper,12pt]{article}

\usepackage{amsfonts, amsmath}
\usepackage{exsheets}
\usepackage[a4paper, margin=1in]{geometry}

\newcommand\Blank[2][.33\linewidth]{%
  % add strut to allow for handwriting
  \rule{0pt}{4ex}%
  % Prompt
  #2\enspace
  % And the actual line
  \makebox[#1]{\hrulefill}}

\title{Computational Thinkerer Workshop\footnote{held at Think tank 13: 25th June 2:45pm to 4:15pm and 26th June 4:30pm to 6:00pm}\\
Finding your Flip / Checking your IC number
\vspace{-2cm}}
%\author{Melvin Zhang}
\date{}

\begin{document}
\maketitle
\begin{flushright}
  \Blank{Name:}

  \Blank{Date:}
\end{flushright}

Digital information is stored and transmitted as bits, a series of
zeros and ones. Sometimes one or more bits may be flipped due to noise or damage
to the media. How can we detect and fix such errors?

\section{Finding your Flip}
In this magic trick, we ask the audience to introduce an ``error'' and then fix
it without knowing where is the error.
\begin{enumerate}
    \item Ask your partner to setup a 5x5 grid of black and white stones.
    \item Add an additional column and row to make it 6x6.
    \item With your back turned, have your partner change one of the stones.
    \item Turn around and magically point out the stone that has just been flipped.
\end{enumerate}

\begin{question}[skip-below=3\baselineskip]
How can we figure out which stone was changed? Hint: the additional
column and row is important!
\end{question}

\begin{question}[skip-below=3\baselineskip]
Practice the trick in pairs.
\end{question}

\begin{question}[skip-below=3\baselineskip]
What happens if two or more cards are flipped? Is it possible to always detect
that an error has occurred?
\end{question}


\newpage
\section{Checking your IC number}
Ever wondered what is the significance of the letter at the end of your NIRC/FIN
number? It serves the same purpose as the extra bits we added in
``Finding your Flip''.

This letter allows anyone reading your IC number to check
whether it is a valid number by comparing the digits of your number against the
letter. If the two do not match, it means there is an error somewhere.

Suppose your IC number starts with T\textbf{ABCDEFG}
\begin{enumerate}
    \item sum $= 2 * A + 7 * B + 6 * C + 5 * D + 4 * E + 3 * F + 2 * G$
    \item Add 4 to sum if your IC number starts with T or G
    \item Find the remainder when sum is divided by 11
    \item Convert the remainder into a letter using the following table
\end{enumerate}

\begin{tabular}{r||ccccccccccc}
    remainder     & 0 & 1 & 2 & 3 & 4 & 5 & 6 & 7 & 8 & 9 & 10 \\ \hline
    letter (NRIC) & J & Z & I & H & G & F & E & D & C & B & A  \\
    letter (FIN)  & X & W & U & T & R & Q & P & N & M & L & K
\end{tabular}


\begin{question}[skip-below=0\baselineskip]
Compute the letter for the following NRIC numbers:
\begin{itemize}
    \item S0000001? (Yusof bin Ishak, first President of Singapore)
    \item S0000002? (Wee Chong Jin, first Chief Justice of Singapore)
    \item S0000003? (Lee Kuan Yew, first Prime Minister of Singapore)
    \item S0000004? (Kwa Geok Choo, wife of Lee Kuan Yew)
    \item S0000005? (Toh Chin Chye, first Deputy Prime Minister of Singapore)
    \item S0000006? (Goh Keng Swee, second Deputy Prime Minister of Singapore)
    \item S0000007? (S Rajaratnam, first Minister for Culture)
\end{itemize}
\end{question}
\begin{solution}
I, G, E, C, A, Z, H
\end{solution}

\begin{question}[skip-below=3\baselineskip]
Compute the letter for your NRIC/FIN number. Ask your partner for
his/her NRIC/FIN number without the letter, then compute
the letter and see if you can get it right.
\end{question}

\begin{question}[skip-below=3\baselineskip]
If we change only a single number, is it possible to get back the same letter?
\end{question}

\begin{solution}
No.
\end{solution}

\begin{question}[skip-below=3\baselineskip]
If we are allowed to change \emph{two} digits, is it possible to get back the letter?
\end{question}

\begin{solution}
Yes. For example, swapping the first and last digit.
\end{solution}

\end{document}
