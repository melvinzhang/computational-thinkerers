\documentclass[a4paper,12pt]{article}

\usepackage{amsfonts, amsmath}
\usepackage{exsheets}
\usepackage[a4paper, margin=1in]{geometry}
\usepackage{graphicx}

\newcommand\Blank[2][.33\linewidth]{%
  % add strut to allow for handwriting
  \rule{0pt}{4ex}%
  % Prompt
  #2\enspace
  % And the actual line
  \makebox[#1]{\hrulefill}}

\ifdefined\withsolution
    \SetupExSheets{solution/print=true,no-skip-below=true}
\fi

\title{Computational Thinkerer Workshop\\
Flexing your Hexahexaflexagon
\vspace{-2cm}}
%\author{Melvin Zhang}
\date{}

\begin{document}
\maketitle
\begin{flushright}
  \Blank{Name:}

  \Blank{Date:}
\end{flushright}

% flexagon graph: draw the hexagon graph
% extension: ???, find out more

Your hexahexaflexagon (HHF) is a paper model which can be flexed
endlessly. It is called a hexahexaflexagon because its shape is a hexagon and it
has six different faces you can find by flexing it.

It turns out when we draw lines on these faces, there is actually more than six
different faces! To explore the HHF systematically, we need a graph.

\let\thefootnote\relax\footnote{Solution available at \texttt{http://melvinzh.sdf.org/2-FYH-worksheet-soln.pdf}}

% bridging: graphs in everyday, MRT map.  
\section*{Example of a graph}
% figure of part of MRT map
\begin{center}
\includegraphics[scale=0.5]{mrt-graph}

Figure 1: An example of a graph for a section of the MRT network.   
\end{center}

A graph consist of \emph{nodes} and \emph{edges}. In Figure 1, the nodes are the
MRT stations and the edges are the railway lines connecting the stations.
Although Tampines is on two different lines, it is considered a single node.  

\begin{question}[skip-below=3\baselineskip]
How many nodes and edges are in Figure 1?
\end{question}
\begin{solution}
There are 6 nodes and 5 edges in Figure 1. 
\end{solution}

\newpage
\let\thefootnote\relax\footnote{Inspired by \texttt{http://www.cs4fn.org/hexahexaflexagon/}}
\section*{Mapping your hexahexaflexagon}
For your HHF, the nodes are the faces and the edges are the flexing actions.

Your objective is to construct a graph of your HHF and find out how many
different faces it has. Remember that each face is one node and each flex is one
edge.

\begin{question}[skip-below=20\baselineskip]
Draw the graph for your HHF.
\end{question}
\begin{solution}
\includegraphics[scale=0.5]{graph}
\end{solution}

\begin{question}[skip-below=3\baselineskip]
How many different faces does a HHF have?
\end{question}
\begin{solution}
There are 9 possible faces.  
\end{solution}

\begin{question}[skip-below=3\baselineskip]
What is the fewest flexes to get from face 6 to face 5?
\end{question}
\begin{solution}
4 flexes is needed.
\end{solution}

\end{document}
